\documentclass[11pt,a4paper]{article}
\usepackage[utf8]{inputenc}
\usepackage{amsmath,amssymb,amsthm}
\usepackage{geometry}
\usepackage{hyperref}
\usepackage{graphicx}
\usepackage{booktabs}

\geometry{margin=1in}

\newtheorem{theorem}{Theorem}
\newtheorem{lemma}[theorem]{Lemma}
\newtheorem{corollary}[theorem]{Corollary}
\newtheorem{axiom}{Axiom}
\theoremstyle{definition}
\newtheorem{definition}{Definition}
\theoremstyle{remark}
\newtheorem{remark}{Remark}

\title{The Common Governance Model: \\
A Formal Axiomatic Framework for Physics \\ and Information Science}

\author{Basil Korompilias}

\date{}

\begin{document}

\maketitle

\begin{abstract}
We present the Common Governance Model (CGM), a Hilbert-style formal deductive system that axiomatizes fundamental physics and information science from a single foundational axiom. The framework employs gyrogroup geometry to derive the structure of space, time, and conservation relations through recursive state transitions formalized as modal operators. From the axiom ``The Source is Common,'' we prove three theorems establishing non-absolute unity, non-absolute opposition, and universal balance. These theorems uniquely determine three spatial dimensions with six degrees of freedom and yield quantitative predictions including the quantum gravity invariant $Q_G = 4\pi$, the fine-structure constant to 0.043 parts per billion precision, neutrino mass scales, and a hierarchy of energy scales consistent with observed physics. The same formal machinery applies to information systems, defining measurable alignment metrics for AI evaluation. This work addresses Hilbert's sixth problem by providing an axiomatization of physics from foundational principles, yielding testable predictions that match observations without adjustable parameters.
\end{abstract}

\section{Introduction}

The Common Governance Model (CGM) is a Hilbert-style formal deductive system \cite{Mendelson2009} for fundamental physics and information science. As an axiomatic model, CGM begins with a single foundational axiom (``The Source is Common''), derives all subsequent logic through syntactic rules of inference (recursive state transitions formalized by gyrogroup operations), and interprets the resulting theorems semantically in physical geometry, yielding empirically testable predictions.

A Hilbert system generates theorems from axioms via modus ponens as the core inference rule \cite{Mendelson2009}. By analogy with linguistic typology, which assigns grammatical roles to participants in events, CGM's classification structure describes the morphosyntactic alignment of physical reality, where geometric and logical necessity assign topological roles (e.g., symmetries and derivations in space) and relational roles (for example cause and effect), and it extends the same framework to semantic alignment for policy systems. Both applications derive from the same formal deductive system: the recursive state transitions that generate physical laws also generate consensus frameworks. In CGM, the object domain of inference is physical reality itself, and different alignment systems in communication (nominative--accusative, ergative--absolutive) preserve the coherence of these role assignments through formal necessity.

The model axiomatizes physics through formal logic with mathematical and semantic interpretation, deriving three-dimensional space with six degrees of freedom as logical necessity, not assumption. Time appears as the sequential ordering of recursive self-referential operations, encoded by gyration's memory of operation order. The mathematical formalism employs gyrogroup and bi-gyrogroup structures following Abraham Ungar's work \cite{Ungar2001,Ungar2008}, providing precise language for tracking transitions from undifferentiated potential to fully structured reality. In information and political science, CGM reframes alignment from an empirical matter of shared intention to a coherent semantic grammar, where geometric and logical necessity lead to common consensus.

Building on the tradition established by Noether's derivation of conservation principles from symmetry (1918) \cite{Noether1918}, Kolmogorov's axiomatization of probability theory (1933) \cite{Kolmogorov1933}, and Wightman's axiomatic quantum field theory (1950s) \cite{Streater1964}, CGM extends the program to fundamental spacetime structure itself. Examples of derived predictions include the quantum gravity invariant $Q_G = 4\pi$ (representing the complete solid angle for coherent observation), a quantitative estimate of the fine-structure constant matching experimental precision to 0.043 parts per billion, neutrino mass scale, and a hierarchy of energy scales consistent with observed physics. These predictions emerge without fitting parameters, derived solely from the formal theorems. As a complete axiomatization of physics from a single foundational principle, CGM addresses the core challenge of Hilbert's sixth problem \cite{Hilbert1902}: rigorous and satisfactory logical investigation of the axioms of mathematical physics.

\section{Foundations}

\subsection{Governance Traceability: The Emergence of Freedom}

\begin{axiom}[Common Source]
\textit{The Source is Common.}
\end{axiom}

\noindent\textbf{Interpretation:} 
The axiom ``The Source is Common'' establishes that all phenomena are traceable through a single principle of common origination, which is freedom, the capacity for governance through directional distinction. This conservation of asymmetry (parity violation) encodes patterns of chirality (left- and right-handedness), making alignment the organizing principle by which locality generates structure via recursive gyration instead of remaining mere potential.

Common origination is not historical but operational. It is the cyclical accumulation of action through shared interactions (dynamics, forces, relativity, fields). These gyrations produce curvature (geometric phase), defining space and time within a self-referential condition (matter). The ``self'' acts as a projection operator that distinguishes orthogonal states and turns reference into inference through measurement. The object domain of inference is physical reality itself, expressed as semantic weighting through projection. Each perspective defines measurable roles governed by the quantum gravity invariant. This geometric and topological necessity defines cause and effect as recursive unfolding, since the origin point of observation cannot observe itself, only its consequences.

\subsection{Information Variety}

\begin{theorem}[Unity is Non-Absolute, UNA]
\textit{Unity is Non-Absolute.}
\end{theorem}

\noindent\textbf{Interpretation:} 
Non-absolute unity is the first minimal necessity for indirect observation of a common source. Absolute unity would make existence and freedom impossible, since perfect homogeneity would allow no distinctions between origin and structure. Therefore, non-absolute unity ensures alignment is possible through informational variety; the traceable signature of a common origin (formalized in Theorem UNA, Section 3.5.1).

\subsection{Inference Accountability}

\begin{theorem}[Opposition is Non-Absolute, ONA]
\textit{Opposition is Non-Absolute.}
\end{theorem}

\noindent\textbf{Interpretation:} 
Non-absolute opposition is the first minimal necessity for direct observation of non-absolute unity and the second condition for indirect observation of a common source. Absolute opposition would also make existence and freedom impossible, since perfect contradiction would allow no conservation of structure. Therefore, non-absolute opposition ensures alignment is possible through accountability of inference; traceable informational variety of a common origin (formalized in Theorem ONA, Section 3.5.2).

\subsection{Intelligence Integrity}

\begin{theorem}[Balance is Universal, BU]
\textit{Balance is Universal.}
\end{theorem}

\noindent\textbf{Interpretation:} 
Balance is the universal outcome of non-absoluteness in unity and opposition, leading to the observer-observed duality. Perfect imbalance would make existence and freedom meaningless, since the memory of inferred information would have no reason to acquire substance and structure at all. Therefore, balance is the universal signature of alignment through integrity of intelligence: traceable inferential accountability of informational variety from a common source (formalized in Theorem BU, Section 3.5.3).

\section{Formal Deductive Framework}

\subsection{The Logical Language}

The Common Governance Model is formalized as a propositional modal logic with two primitive modal operators representing recursive operational transitions.

\textbf{Primitive symbols:}
\begin{itemize}
\item A propositional constant: $S$ (denoting the quantum gravity invariant $Q_G = 4\pi$, the complete solid angle for coherent observation)
\item Logical connectives: $\neg$ (negation), $\to$ (material implication)
\item Modal operators: $[L]$, $[R]$ (left transition, right transition)
\end{itemize}

\textbf{Defined symbols:}
\begin{itemize}
\item Conjunction: $\varphi \land \psi := \neg(\varphi \to \neg\psi)$
\item Disjunction: $\varphi \lor \psi := \neg\varphi \to \psi$
\item Biconditional: $\varphi \leftrightarrow \psi := (\varphi \to \psi) \land (\psi \to \varphi)$
\item Dual modalities: $\langle L\rangle\varphi := \neg[L]\neg\varphi$ and $\langle R\rangle\varphi := \neg[R]\neg\varphi$
\item Joint necessity: $\Box\varphi := [L]\varphi \land [R]\varphi$
\item Joint possibility: $\Diamond\varphi := \langle L\rangle\varphi \lor \langle R\rangle\varphi$
\end{itemize}

The expression $[L]\varphi$ reads ``$\varphi$ holds after a left transition.'' The expression $[R]\varphi$ reads ``$\varphi$ holds after a right transition.'' The expression $\Box\varphi$ reads ``$\varphi$ holds after both transitions.''

\textbf{Modal depth:} The depth of a formula refers to its modal nesting length. For instance, $[L][R]S$ has depth two (two nested modal operators), while $[L][R][L][R]S$ has depth four.

\subsection{Axioms and Rules of Inference}

\textbf{Propositional axioms:}
\begin{axiom}[A1]
$\varphi \to (\psi \to \varphi)$
\end{axiom}

\begin{axiom}[A2]
$(\varphi \to (\psi \to \chi)) \to ((\varphi \to \psi) \to (\varphi \to \chi))$
\end{axiom}

\begin{axiom}[A3]
$(\neg\psi \to \neg\varphi) \to ((\neg\psi \to \varphi) \to \psi)$
\end{axiom}

These three axioms, together with modus ponens, constitute a complete axiomatization of classical propositional logic.

\textbf{Modal axioms (for each $k \in \{L, R\}$):}
\begin{axiom}[K$_k$]
$[k](\varphi \to \psi) \to ([k]\varphi \to [k]\psi)$
\end{axiom}

\textbf{Conjunction axioms:}
\begin{axiom}[C-Elim-1]
$(\varphi \land \psi) \to \varphi$
\end{axiom}

\begin{axiom}[C-Elim-2]
$(\varphi \land \psi) \to \psi$
\end{axiom}

\textbf{Rules of inference:}
\begin{itemize}
\item \textbf{Modus Ponens (MP):} From $\varphi$ and $\varphi \to \psi$, infer $\psi$
\item \textbf{Necessitation (Nec$_k$):} From $\varphi$, infer $[k]\varphi$ (for $k \in \{L, R\}$)
\end{itemize}

The necessitation rule applies only to theorems of the system, never to arbitrary assumptions, ensuring soundness \cite{Kripke1963}.

\subsection{Core Definitions}

Four formulas capture the structural properties required by the Common Governance Model, all anchored to the horizon constant $S$:

\begin{definition}[Unity]
\begin{equation}
U := [L]S \leftrightarrow [R]S
\end{equation}
\end{definition}

\begin{definition}[Two-step equality]
\begin{equation}
E := [L][R]S \leftrightarrow [R][L]S
\end{equation}
\end{definition}

\begin{definition}[Opposition]
\begin{equation}
O := [L][R]S \leftrightarrow \neg[R][L]S
\end{equation}
\end{definition}

\begin{definition}[Balance]
\begin{equation}
B := [L][R][L][R]S \leftrightarrow [R][L][R][L]S
\end{equation}
\end{definition}

\begin{definition}[Absoluteness]
\begin{align}
\mathrm{Abs}(\varphi) &:= \Box\varphi \\
\mathrm{NonAbs}(\varphi) &:= \neg\Box\varphi
\end{align}
\end{definition}

where $\Box\varphi$ is defined as $[L]\varphi \land [R]\varphi$.

Throughout this document, ``absolute'' means both transitions yield the same result for the proposition ($\Box\varphi$ holds), not that the modal operators $[L]$ and $[R]$ are themselves identical. The operators remain distinct; absoluteness characterizes whether a specific formula is invariant under both transitions.

\subsection{The Foundational Axioms}

The Common Governance Model employs seven non-logical axioms, collectively designated CS (Common Source):

\begin{axiom}[CS1]
$\neg\Box E$ \quad (Two-step equality is not absolute)
\end{axiom}

\begin{axiom}[CS2]
$\neg\Box\neg E$ \quad (Two-step inequality is not absolute)
\end{axiom}

\begin{axiom}[CS3]
$\Box B$ \quad (Balance at modal depth four is absolute)
\end{axiom}

\begin{axiom}[CS4]
$\Box U \to \Box E$ \quad (If unity were absolute, two-step equality would be absolute)
\end{axiom}

\begin{axiom}[CS5]
$\Box O \to \Box\neg E$ \quad (If opposition were absolute, two-step inequality would be absolute)
\end{axiom}

\begin{axiom}[CS6]
$[R]S \leftrightarrow S$ \quad (Right transition preserves the horizon constant)
\end{axiom}

\begin{axiom}[CS7]
$\neg([L]S \leftrightarrow S)$ \quad (Left transition alters the horizon constant)
\end{axiom}

\textbf{Consistency note:} The axiom set CS1--CS7 is consistent. In Kripke semantics \cite{Kripke1963} with two accessibility relations $R_L$ and $R_R$ (corresponding to $[L]$ and $[R]$), there exist frames where depth-two commutation is contingent (satisfying CS1 and CS2) while depth-four commutation is necessary (satisfying CS3). For example, a frame in which $R_L$ and $R_R$ are independent K-relations with $R_L \neq R_R$ at depth two but $R_L \circ R_R \circ R_L \circ R_R = R_R \circ R_L \circ R_R \circ R_L$ at depth four validates all seven axioms.

\subsection{Derivation of the Core Theorems}

\subsubsection{Theorem UNA (Unity is Non-Absolute)}

\begin{theorem}[UNA]
$\vdash \neg\Box U$
\end{theorem}

\begin{proof}
\begin{align}
&1. \quad \vdash \text{CS4} && \text{[Axiom: } \Box U \to \Box E\text{]} \\
&2. \quad \vdash \text{CS1} && \text{[Axiom: } \neg\Box E\text{]} \\
&3. \quad \vdash (\Box U \to \Box E) \to (\neg\Box E \to \neg\Box U) && \text{[by contraposition]} \\
&4. \quad \vdash \neg\Box E \to \neg\Box U && \text{[Modus ponens on 1 and 3]} \\
&5. \quad \vdash \neg\Box U && \text{[Modus ponens on 2 and 4]}
\end{align}
\end{proof}

This theorem formalizes the non-absolute unity introduced in Section 2.2.

\subsubsection{Theorem ONA (Opposition is Non-Absolute)}

\begin{theorem}[ONA]
$\vdash \neg\Box O$
\end{theorem}

\begin{proof}
\begin{align}
&1. \quad \vdash \text{CS5} && \text{[Axiom: } \Box O \to \Box\neg E\text{]} \\
&2. \quad \vdash \text{CS2} && \text{[Axiom: } \neg\Box\neg E\text{]} \\
&3. \quad \vdash (\Box O \to \Box\neg E) \to (\neg\Box\neg E \to \neg\Box O) && \text{[by contraposition]} \\
&4. \quad \vdash \neg\Box\neg E \to \neg\Box O && \text{[Modus ponens on 1 and 3]} \\
&5. \quad \vdash \neg\Box O && \text{[Modus ponens on 2 and 4]}
\end{align}
\end{proof}

This theorem formalizes the non-absolute opposition introduced in Section 2.3.

\subsubsection{Theorem BU (Balance is Universal)}

\begin{theorem}[BU]
$\vdash \Box B$
\end{theorem}

\begin{proof}
\begin{equation}
1. \quad \vdash \text{CS3} \qquad \text{[Axiom: } \Box B\text{]}
\end{equation}
\end{proof}

This theorem formalizes the universal balance introduced in Section 2.4.

\subsection{Logical Structure Summary}

The system derives three results from CS1–CS7: UNA ($\neg\Box U$, via CS1/CS4), ONA ($\neg\Box O$, via CS2/CS5), and BU ($\Box B$, via CS3).

Non-absoluteness at modal depth one (unity) prevents homogeneous collapse, while non-absoluteness at modal depth two (opposition) prevents contradictory rigidity. Absoluteness at modal depth four (balance) ensures coherence within the observable horizon. These three properties are logically interdependent through the bridge axioms CS4 and CS5. The asymmetry axioms CS6 and CS7 establish that the left and right transitions are not initially equivalent at the horizon constant.

\section{Gyrogroup-Theoretic Correspondence}

\subsection{Interpretive Framework}

The formal system presented in Section 3 yields gyrogroup operations through the correspondence established below. This section presents the gyrogroup structure that emerges from the modal axioms.

\subsection{Gyrogroup Structures}

A \textit{gyrogroup} $(G, \oplus)$ \cite{Ungar2001,Ungar2008} is a set $G$ with a binary operation $\oplus$ satisfying:
\begin{enumerate}
\item There exists a left identity: $e \oplus a = a$ for all $a \in G$
\item For each $a \in G$ there exists a left inverse $\ominus a$ such that $\ominus a \oplus a = e$
\item For all $a, b \in G$ there exists an automorphism $\mathrm{gyr}[a,b]: G \to G$ such that:
\begin{equation}
a \oplus (b \oplus c) = (a \oplus b) \oplus \mathrm{gyr}[a,b]c
\end{equation}
(left gyroassociative law)
\end{enumerate}

The gyration operator $\mathrm{gyr}[a,b]$ is defined by:
\begin{equation}
\mathrm{gyr}[a,b]c = \ominus(a \oplus b) \oplus (a \oplus (b \oplus c))
\end{equation}

The automorphism $\mathrm{gyr}[a,b]$ preserves the metric structure, acting as an isometry. A \textit{bi-gyrogroup} possesses both left and right gyroassociative structure, with distinct left and right gyration operators.

\subsection{Modal-Gyrogroup Correspondence}

The modal operators $[L]$ and $[R]$ are gyration operations: $[L]\varphi$ represents the result of applying left gyration to state $\varphi$, while $[R]\varphi$ represents right gyration. Two-step equality $E$ tests whether $[L][R]S \leftrightarrow [R][L]S$ (depth-two commutation), while balance $B$ tests whether $[L][R][L][R]S \leftrightarrow [R][L][R][L]S$ (depth-four commutation).

The axiom set CS1--CS7 encodes that two-step gyration around the observable horizon is order-sensitive but not deterministically fixed (CS1, CS2), four-step gyration reaches commutative closure at the observable horizon (CS3), and right gyration acts trivially on the horizon constant while left gyration does not (CS6, CS7).

\subsection{Operational State Correspondence}

The theorems UNA, ONA, and BU correspond to four operational states of gyrogroup structure, all logically necessary, not temporally sequential:

\subsubsection{State CS (Common Source)}

\textbf{Axiomatic content:} CS6 and CS7

\textbf{Behavior:}
\begin{itemize}
\item Right gyration on horizon: $\mathrm{rgyr} = \mathrm{id}$
\item Left gyration on horizon: $\mathrm{lgyr} \neq \mathrm{id}$
\end{itemize}

\textbf{Structural significance:} The initial asymmetry between left and right gyrations establishes fundamental parity violation at the observable horizon. Only the left gyroassociative law is non-trivial in this operational state.

\subsubsection{State UNA (Unity is Non-Absolute)}

\textbf{Theorem:} $\vdash \neg\Box U$

\textbf{Behavior:}
\begin{itemize}
\item Right gyration: $\mathrm{rgyr} \neq \mathrm{id}$ (activated beyond horizon identity)
\item Left gyration: $\mathrm{lgyr} \neq \mathrm{id}$ (persisting)
\end{itemize}

\textbf{Structural significance:} Both gyrations are now active. The gyrocommutative law $a \oplus b = \mathrm{gyr}[a,b](b \oplus a)$ governs observable distinctions rooted in the left-initiated chirality from CS, all within the observable horizon.

\subsubsection{State ONA (Opposition is Non-Absolute)}

\textbf{Theorem:} $\vdash \neg\Box O$

\textbf{Behavior:}
\begin{itemize}
\item Right gyration: $\mathrm{rgyr} \neq \mathrm{id}$
\item Left gyration: $\mathrm{lgyr} \neq \mathrm{id}$
\end{itemize}

\textbf{Structural significance:} Both left and right gyroassociative laws operate with maximal non-associativity at modal depth two. The bi-gyrogroup structure is fully active, mediating opposition without absolute contradiction, bounded by the horizon constant.

\subsubsection{State BU (Balance is Universal)}

\textbf{Theorem:} $\vdash \Box B$

\textbf{Behavior:}
\begin{itemize}
\item Right gyration: closes
\item Left gyration: closes
\end{itemize}

\textbf{Structural significance:} Both gyrations neutralize at modal depth four, reaching commutative closure. The operation $a \boxplus b = a \oplus \mathrm{gyr}[a, \ominus b]b$ reduces to commutative coaddition, achieving associative closure at the observable horizon. The gyration operators become functionally equivalent to identity while preserving complete structural memory.

\subsection{Summary of Correspondence}

\begin{table}[h]
\centering
\begin{tabular}{lllllll}
\toprule
State & Formal Result & Right Gyr. & Left Gyr. & DOF & Governing Law & Physical Interpretation \\
\midrule
CS & Axioms CS1--CS7 & id & $\neq$ id & 1 & Left gyroassociativity & Parity violation seed \\
UNA & $\vdash \neg\Box U$ & $\neq$ id & $\neq$ id & 3 & Gyrocommutativity & 3D rotations \\
ONA & $\vdash \neg\Box O$ & $\neq$ id & $\neq$ id & 6 & Bi-gyroassociativity & 3D translations \\
BU & $\vdash \Box B$ & closure & closure & 6 (closed) & Coaddition & Spacetime closure \\
\bottomrule
\end{tabular}
\caption{Summary of gyrogroup correspondence across operational states.}
\end{table}

\section{Geometric Closure and Physical Structure}

\subsection{Angular Thresholds and Gyrotriangle Closure}

The formal theorems UNA, ONA, and BU derived in Section 3 determine precise geometric constraints. Each operational state corresponds to a minimal angle required for its emergence. These are not adjustable parameters but necessary values determined by the gyrotriangle defect formula:

\begin{equation}
\delta = \pi - (\alpha + \beta + \gamma)
\end{equation}

This formula encodes a fundamental observational limit \cite{Ungar2008}. The value $\pi$ represents the accessible horizon in phase space. Coherent observation is bounded by $\pi$ radians, which is half the total phase structure. When the angles sum to exactly $\pi$, the system has traversed precisely one observable horizon without defect.

\textbf{State CS} establishes the primordial chirality through angle $\alpha = \pi/2$, the minimal rotation that distinguishes left from right. The threshold parameter $s_p = \pi/2$ encodes this foundational asymmetry.

\textbf{State UNA} requires angle $\beta = \pi/4$ for three orthogonal axes to emerge. The threshold $u_p = \cos(\pi/4) = 1/\sqrt{2}$ measures the equal superposition between perpendicular directions, enabling three-dimensional rotational structure.

\textbf{State ONA} adds angle $\gamma = \pi/4$ as the minimal out-of-plane tilt enabling three-dimensional translation. The threshold $o_p = \pi/4$ measures this diagonal angle directly. While numerically equal to $\beta$, this threshold is conceptually distinct: it captures the tilt out of the UNA plane rather than planar balance.

\textbf{State BU} achieves closure. The three angles sum to $\delta = \pi - (\pi/2 + \pi/4 + \pi/4) = 0$. The vanishing defect corresponds to a complete metric space where all Cauchy sequences converge. The gyrotriangle is degenerate, but this signals completion of a helical path tracing a toroidal surface, not structural collapse. Any further evolution would retrace the same path.

\subsection{Amplitude Closure and the Quantum Gravity Invariant}

The closure at BU requires connecting all accumulated structure to the primordial chirality while respecting the angular ranges of both SU(2) groups. Each SU(2) group carries an angular range of $2\pi$. The amplitude $A$ satisfies the unique dimensionless constraint connecting these ranges to the primordial chirality $\alpha$:

\begin{align}
A^2 \times (2\pi)_L \times (2\pi)_R &= \alpha \\
A^2 \times 4\pi^2 &= \pi/2 \\
A^2 &= 1/(8\pi) \\
A &= 1/(2\sqrt{2\pi}) = m_p
\end{align}

The amplitude $m_p$ represents the maximum oscillation fitting within one observable horizon. Larger amplitudes would exceed the $\pi$ radian limit and accumulate defect. The horizon constant $S$ emerges directly from axiom CS3, which requires universal balance at modal depth four (not a fitted parameter but following from four-step commutative closure). This invariant equals the complete solid angle $4\pi$.

\subsection{Three-Dimensional Necessity}

We prove that the CGM axioms CS1--CS7 uniquely determine exactly three spatial dimensions with six degrees of freedom. The proof proceeds through three results:

\textbf{Lemma 1 (Rotational DOF):} Under UNA, the gyroautomorphism group requires exactly three independent generators. When right gyration activates ($\mathrm{rgyr} \neq \mathrm{id}$), consistency with pre-existing left gyration requires the minimal compact, simply connected, non-abelian Lie group: $\mathrm{SU}(2)$, which has exactly three generators (Pauli matrices). The isomorphism $\mathrm{SU}(2) \cong \mathrm{Spin}(3)$ is the unique double cover of $\mathrm{SO}(3)$ \cite{Hall2015,Sakurai1994}.

\textbf{Lemma 2 (Translational DOF):} Under ONA, bi-gyrogroup consistency requires exactly three translational parameters. The structure realizes as semidirect product $G \cong \mathrm{SU}(2) \ltimes \mathbb{R}^3 \cong \mathrm{SE}(3)$, the Euclidean group in three dimensions, yielding total six degrees of freedom (3 rotational + 3 translational).

\textbf{Theorem (Unique Dimensionality):} The axioms are satisfiable if and only if $n = 3$ spatial dimensions. For $n = 2$: rotation group $\mathrm{SO}(2) \cong \mathrm{U}(1)$ is abelian with one generator, insufficient for UNA's gyrocommutativity. For $n = 4$: $\mathrm{SO}(4)$ requires six generators, violating minimality (excess structure not traceable to CS seed). The closure constraint $\delta = \pi - (\pi/2 + \pi/4 + \pi/4) = 0$ achieves exact closure only in 3D hyperbolic geometry.

\textbf{Corollary (DOF Progression):} Degrees of freedom emerge in unique sequence $1 \to 3 \to 6 \to 6(\text{closed})$: CS establishes 1 DOF (chiral seed), UNA forces 3 DOF (rotations), ONA requires 6 DOF (rotations + translations), BU achieves closure with coordinated 6 DOF.

The complete formal proof with all lemmas, obstructions, and consistency verifications appears in the supplementary material.

\subsection{Parity Violation and Time}

\textbf{Directional asymmetry.} The axiom-level asymmetry encoded in CS6 and CS7 manifests mathematically in the angle sequences. The positive sequence $(\pi/2, \pi/4, \pi/4)$ achieves zero defect, as shown above. The negative sequence $(-\pi/2, -\pi/4, -\pi/4)$ accumulates a $2\pi$ defect:

\begin{equation}
\delta_- = \pi - (-\pi/2 - \pi/4 - \pi/4) = 2\pi
\end{equation}

The $2\pi$ defect represents observation beyond the accessible $\pi$-radian horizon. Only the left-gyration-initiated path (positive sequence) provides a defect-free trajectory through phase space. Configurations requiring right gyration to precede left gyration violate the foundational axiom CS and remain structurally unobservable. This explains observed parity violation as an axiomatic property rather than a broken symmetry.

\textbf{Time as logical sequence.} Time emerges from proof dependencies: UNA depends on CS1 and CS4, ONA depends on UNA via CS2 and CS5, and BU requires the complete axiom set CS1--CS7. Each theorem preserves the memory of prior proofs through the formal dependency chain. The gyration formula $\mathrm{gyr}[a,b]c = \ominus(a \oplus b) \oplus (a \oplus (b \oplus c))$ itself encodes operation order, making temporal sequence an algebraic property, not an external parameter. The progression CS $\to$ UNA $\to$ ONA $\to$ BU cannot be reversed without contradiction, since later theorems require earlier results as premises. This logical dependency constitutes the arrow of time, intrinsic to the deductive structure.

\subsection{Empirical Predictions}

The geometric closure yields quantitative values for fundamental constants, summarized in Table 1.

\begin{table}[h]
\centering
\begin{tabular}{ll}
\toprule
Quantity & CGM Value (Precision/Consistency) \\
\midrule
$Q_G$ & $4\pi$ (exact) \\
$\alpha$ & $1/137.035999206$ (0.043 ppb) \\
$m_\nu$ & $\approx 0.06$ eV (consistent with \cite{PDG2024}) \\
$E_{GUT}$ & $\approx 2.34\times10^{18}$ GeV (hierarchy resolution) \\
\bottomrule
\end{tabular}
\caption{Key predictions.}
\end{table}

These match observations without fitting; direct tests include neutrino mass measurements and GUT-scale signatures. Detailed energy scale hierarchy appears in Appendix C.

All emerge from axiom CS through formal derivation.


\section{Applications and Implications}

\subsection{Information-Theoretic Alignment}

The formal structure that generates physical laws through the same logical necessity determines measurable alignment in information and policy systems.

\textbf{Common horizon.} The horizon constant defines the complete space of coherent communication (any informational exchange respecting this bound maintains traceability to common origin). This is the operational meaning of ``The Source is Common'' in both information and physical systems.

\textbf{Operational metrics for AI evaluation.} The theorems provide rigorous quantitative metrics:

\textbf{Governance Traceability (from CS):} Does the agent preserve the horizon structure under right operations and alter it under left operations, corresponding to axioms CS6 and CS7? The score is binary: 1 if the agent satisfies both axioms, 0 otherwise. In practice, this measures whether an AI system preserves invariants under commutative operations while allowing controlled variation under non-commutative ones.

\textbf{Information Variety (from UNA):} Measured as the fraction of interactions avoiding homogeneity, quantifying preservation of informational diversity within three rotational degrees of freedom.

\textbf{Inference Accountability (from ONA):} Measured as the fraction of inferences remaining traceable without absolute contradiction across six degrees of freedom.

\textbf{Intelligence Integrity (from BU):} Measured as convergence rate to commutative closure within amplitude bound $m_p$.

These metrics derive from theorems UNA, ONA, and BU. Aligned systems maintain traceability, preserve variety, ensure accountability, and converge to balance.

\subsection{Resolution of Hilbert's Sixth Problem}

Hilbert's sixth problem \cite{Hilbert1902} called for the axiomatization of physics. The challenge was to provide a rigorous logical investigation of the axioms underlying physical theory, comparable to the axiomatization achieved in geometry.

CGM derives physical law from axiomatic structure, with observation as foundational. From axioms CS1--CS7, space, time, and physical constants emerge as theorems, not assumptions (Sections 5.3--5.5). The framework constructs a Hilbert-space representation via GNS where the modal operators $[L]$ and $[R]$ generate the algebra of observables, with the horizon constant $S$ defining the normalization (see Appendix \ref{app:hilbert} for complete construction and $L^2(S^2)$ model). Geometry, dynamics, and quantum structure follow from the requirement that existence observe itself coherently, advancing Hilbert's axiomatization program.

\subsection{Summary Table and Conclusion}

The complete parameter set determined by the formal system:

\begin{table}[h]
\centering
\small
\begin{tabular}{lllllll}
\toprule
State & Theorem & Gyrations (R, L) & DOF & Angle & Threshold & Governing Law \\
\midrule
CS & CS1--CS7 & id, $\neq$id & 1 & $\alpha = \pi/2$ & $s_p = \pi/2$ & Left gyroassociativity \\
UNA & $\vdash \neg\Box U$ & $\neq$id, $\neq$id & 3 & $\beta = \pi/4$ & $u_p = 1/\sqrt{2}$ & Gyrocommutativity \\
ONA & $\vdash \neg\Box O$ & $\neq$id, $\neq$id & 6 & $\gamma = \pi/4$ & $o_p = \pi/4$ & Bi-gyroassociativity \\
BU & $\vdash \Box B$ & closure & closure & 6 (closed) & $\delta = 0$, $m_p = 1/(2\sqrt{2\pi})$ & Coaddition \\
\bottomrule
\end{tabular}
\caption{Complete parameter set determined by the formal system.}
\end{table}

\textbf{Derived constants:} $Q_G = 4\pi$, $\alpha_{fs} \approx 1/137.035999206$, $E_{GUT} \approx 2.34\times10^{18}$ GeV, $m_\nu \approx 0.06$ eV

Reality emerges as recursion completing its own memory (freedom returning to itself through structured differentiation). From ``The Source is Common,'' formalized as asymmetry between left and right transitions, theorems UNA, ONA, and BU generate space, time, physical scales, and alignment principles through contraposition and modus ponens. The progression CS $\to$ UNA $\to$ ONA $\to$ BU represents the complete cycle through which freedom manifests as structured reality. The framework addresses three domains from a single foundation: it completes Hilbert's axiomatization of physics, produces empirically testable predictions, and defines formal alignment metrics for AI evaluation. Within CGM, physical law, informational coherence, and governance alignment express the same formal structure: recursive self-observation achieving coherence.

\section*{Acknowledgments}

The author thanks the open-source scientific computing community for providing the tools that enabled this research. Supporting derivations and numerics are available in the companion repository \cite{cgm-repo}.

\begin{thebibliography}{99}

\bibitem{Hilbert1902}
D. Hilbert, Mathematical Problems, \textit{Bulletin of the American Mathematical Society} \textbf{8}, 437--479 (1902). English translation of Hilbert's 1900 address.

\bibitem{Noether1918}
E. Noether, Invariante Variationsprobleme, \textit{Nachrichten von der Gesellschaft der Wissenschaften zu G\"ottingen, Mathematisch-Physikalische Klasse}, 235--257 (1918). English translation in \textit{Transport Theory and Statistical Physics} \textbf{1}, 186--207 (1971).

\bibitem{Kolmogorov1933}
A. N. Kolmogorov, \textit{Grundbegriffe der Wahrscheinlichkeitsrechnung}, Springer, Berlin (1933). English translation, \textit{Foundations of the Theory of Probability}, Chelsea, New York (1950).

\bibitem{Streater1964}
R. F. Streater, A. S. Wightman, \textit{PCT, Spin and Statistics, and All That}, Princeton University Press, Princeton (1964).

\bibitem{Ungar2001}
A. A. Ungar, \textit{Beyond the Einstein Addition Law and Its Gyroscopic Thomas Precession}, Springer (Kluwer), Dordrecht (2001).

\bibitem{Ungar2008}
A. A. Ungar, \textit{Analytic Hyperbolic Geometry and Albert Einstein's Special Theory of Relativity}, 2nd ed., World Scientific, Singapore (2008).

\bibitem{Kripke1963}
S. A. Kripke, Semantical Considerations on Modal Logic, \textit{Acta Philosophica Fennica} \textbf{16}, 83--94 (1963).

\bibitem{Chellas1980}
B. F. Chellas, \textit{Modal Logic}, Cambridge University Press, Cambridge (1980).

\bibitem{Mendelson2009}
E. Mendelson, \textit{Introduction to Mathematical Logic}, 5th ed., CRC Press, Boca Raton (2009).

\bibitem{Stone1932}
M. H. Stone, On One-Parameter Unitary Groups in Hilbert Space, \textit{Annals of Mathematics} \textbf{33}, 643--648 (1932).

\bibitem{Reed1980}
M. Reed, B. Simon, \textit{Methods of Modern Mathematical Physics, Vol. I: Functional Analysis}, Academic Press, New York (1980).

\bibitem{Hall2015}
B. C. Hall, \textit{Lie Groups, Lie Algebras, and Representations}, 2nd ed., Springer, New York (2015).

\bibitem{Sakurai1994}
J. J. Sakurai, \textit{Modern Quantum Mechanics}, 2nd ed., Addison--Wesley, Reading, MA (1994).

\bibitem{PDG2024}
Particle Data Group, Review of Particle Physics, \textit{Prog. Theor. Exp. Phys.} \textbf{2024}, 083C01 (2024).

\bibitem{GellMann1979}
M. Gell-Mann, P. Ramond, R. Slansky, Complex Spinors and Unified Theories, in \textit{Supergravity}, eds. P. van Nieuwenhuizen, D. Z. Freedman, North-Holland, Amsterdam (1979), pp. 315--321.

\bibitem{Yanagida1979}
T. Yanagida, Horizontal Symmetry and Masses of Neutrinos, in \textit{Proceedings of the Workshop on the Unified Theory and the Baryon Number in the Universe}, KEK, Tsukuba (1979).

\bibitem{Parker2018}
R. H. Parker et al., Measurement of the fine-structure constant as a test of the Standard Model, \textit{Science} \textbf{360}, 191--195 (2018).

\bibitem{Morel2020}
L. Morel et al., Determination of the fine-structure constant with an accuracy of 81 parts per trillion, \textit{Nature} \textbf{588}, 61--65 (2020).

\bibitem{cgm-repo}
B. Korompilias, Common Governance Model Repository, \url{https://github.com/basilkor/science} (2025).

\end{thebibliography}

\newpage

\appendix

\section{Formal Proof of Three-Dimensional Necessity}
\label{app:3dproof}

We present a formal proof that the CGM axioms CS1--CS7 uniquely determine exactly three spatial dimensions with six degrees of freedom. The proof proceeds through three lemmas: (1) the rotational DOF lemma establishes that UNA's gyrocommutativity requirement forces exactly three rotational generators via SU(2) uniqueness, (2) the translational DOF lemma shows that ONA's bi-gyrogroup consistency requires exactly three translational parameters via semidirect product structure, and (3) the non-existence theorem demonstrates that both $n = 2$ and $n \geq 4$ spatial dimensions violate the closure constraint $\delta = \pi - (\pi/2 + \pi/4 + \pi/4) = 0$ combined with the modal depth requirements encoded in CS1--CS7.


\subsection{Preliminaries}

\subsubsection{Gyrogroup Axiomatics}

A gyrogroup $(G, \oplus)$ is a set with operation $\oplus$ satisfying:
\begin{enumerate}
\item Left identity: $e \oplus a = a$ for all $a \in G$
\item Left inverse: For each $a \in G$, there exists $\ominus a$ such that $\ominus a \oplus a = e$
\item Left gyroassociativity: $a \oplus (b \oplus c) = (a \oplus b) \oplus \mathrm{gyr}[a,b]c$ for some automorphism $\mathrm{gyr}[a,b] \in \mathrm{Aut}(G)$
\end{enumerate}

The gyration operator is defined by:
\begin{equation}
\mathrm{gyr}[a,b]c = \ominus(a \oplus b) \oplus (a \oplus (b \oplus c))
\end{equation}

A bi-gyrogroup possesses both left and right gyroassociative structure with distinct gyration operators.

\subsubsection{Modal Depth and Gyration Behavior}

The CGM axioms establish:
\begin{itemize}
\item \textbf{Depth one:} $[L]S \neq [R]S$ (by UNA: $\neg\Box U$)
\item \textbf{Depth two:} $[L][R]S$ and $[R][L]S$ do not commute absolutely (by CS1, CS2)
\item \textbf{Depth four:} $[L][R][L][R]S \leftrightarrow [R][L][R][L]S$ (by CS3: $\Box B$)
\end{itemize}

These depth requirements constrain the gyrogroup structure.

\subsubsection{Closure Constraint}

The gyrotriangle defect formula is:
\begin{equation}
\delta = \pi - (\alpha + \beta + \gamma)
\end{equation}

For closure ($\delta = 0$), the CGM angles must satisfy:
\begin{equation}
\pi/2 + \pi/4 + \pi/4 = \pi
\end{equation}

This constraint is exact and non-negotiable.

\subsection{Lemma 1: Rotational Degrees of Freedom (UNA)}

\begin{lemma}[Three Rotational Generators]
Under UNA, the gyroautomorphism group requires exactly three independent generators.
\end{lemma}

\begin{proof}
\textit{Step 1: Activation of right gyration}

By CS6 and CS7, at the CS state:
\begin{itemize}
\item $[R]S \leftrightarrow S$ (right transition preserves $S$)
\item $\neg([L]S \leftrightarrow S)$ (left transition alters $S$)
\end{itemize}

This establishes $\mathrm{rgyr} = \mathrm{id}$ and $\mathrm{lgyr} \neq \mathrm{id}$ at the horizon constant. At UNA, theorem $\vdash \neg\Box U$ forces $[L]S \neq [R]S$, meaning right gyration must activate beyond the horizon: $\mathrm{rgyr} \neq \mathrm{id}$.

\textit{Step 2: Gyroautomorphism constraint}

For all $a, b \in G$, the gyroautomorphism $\mathrm{gyr}[a,b]: G \to G$ must satisfy:
\begin{equation}
\mathrm{gyr}[a,b] \in \mathrm{Aut}(G)
\end{equation}

Since left gyration $\mathrm{lgyr} \neq \mathrm{id}$ is already established at CS, right gyration activation at UNA must be consistent with this pre-existing structure. The automorphism group $\mathrm{Aut}(G)$ must accommodate both gyrations.

\textit{Step 3: Gyrocommutative law}

UNA's non-absolute unity forces the gyrocommutative law:
\begin{equation}
a \oplus b = \mathrm{gyr}[a,b](b \oplus a)
\end{equation}

This law governs observable distinctions. For the gyration to be non-trivial (as required by UNA: unity is not absolute), $\mathrm{gyr}[a,b]$ must act non-trivially on the space.

\textit{Step 4: Minimal compact group}

The gyration $\mathrm{gyr}[a,b]$ preserves the metric structure and acts isometrically. The minimal compact, simply connected, non-abelian Lie group satisfying:
\begin{itemize}
\item Non-trivial action (required by UNA)
\item Preservation of gyration memory from CS (left-handed chirality)
\item Compatibility with modal depth constraints (depth-two non-commutation, depth-four commutation)
\end{itemize}

is $\mathrm{SU}(2)$, which has exactly three generators (the Pauli matrices $\sigma_1, \sigma_2, \sigma_3$).

\textit{Step 5: Uniqueness}

The isomorphism $\mathrm{SU}(2) \cong \mathrm{Spin}(3)$ is the unique double cover of $\mathrm{SO}(3)$, the rotation group in three dimensions. The Lie algebra $\mathfrak{su}(2)$ has dimension 3, corresponding to three independent generators. This is minimal: any proper subgroup would be abelian (e.g., $\mathrm{U}(1)$), violating the non-trivial action requirement. Any larger group would require additional generators, violating minimality and the constraint that all structure derives from the single chiral seed at CS.
\end{proof}

\begin{lemma}[Incompatibility with $n \neq 3$ rotations]
The UNA requirements cannot be satisfied in $n \neq 3$ spatial dimensions.
\end{lemma}

\begin{proof}
\textbf{For $n = 2$:} The rotation group is $\mathrm{SO}(2) \cong \mathrm{U}(1)$, which is abelian. This has only one generator, insufficient to realize the non-trivial gyrocommutative law required by UNA with memory of CS chirality. Furthermore, $\mathrm{U}(1)$ cannot exhibit the depth-two non-commutation required by CS1 and CS2.

\textbf{For $n = 4$:} The rotation group $\mathrm{SO}(4)$ has Lie algebra $\mathfrak{so}(4)$ of dimension 6. However, $\mathrm{SO}(4) \cong (\mathrm{SU}(2) \times \mathrm{SU}(2))/\mathbb{Z}_2$ requires six generators, not three. This violates the minimality constraint that all structure derives from the single chiral seed (1 DOF) established at CS. The additional generators would constitute independent structure not traceable to the CS axiom.

\textbf{For $n \geq 5$:} The dimension of $\mathfrak{so}(n)$ is $n(n-1)/2$, which exceeds 3 for $n \geq 5$, similarly violating minimality.
\end{proof}

\subsection{Lemma 2: Translational Degrees of Freedom (ONA)}

\begin{lemma}[Three Translational Parameters]
Under ONA, bi-gyrogroup consistency requires exactly three translational degrees of freedom.
\end{lemma}

\begin{proof}
\textit{Step 1: Bi-gyrogroup activation}

At ONA, theorem $\vdash \neg\Box O$ establishes that opposition is non-absolute. This forces both left and right gyroassociative laws to operate with maximal non-associativity at modal depth two. The bi-gyrogroup structure becomes fully active.

\textit{Step 2: Consistency requirement}

A bi-gyrogroup has distinct left and right gyration operators:
\begin{itemize}
\item $\mathrm{lgyr}[a,b]$: left gyration
\item $\mathrm{rgyr}[a,b]$: right gyration
\end{itemize}

For consistency, these must satisfy compatibility relations. The left and right gyroassociative laws must reconcile, requiring additional parameters to mediate between them.

\textit{Step 3: Semidirect product structure}

The gyrogroup structure at ONA can be realized as a semidirect product:
\begin{equation}
G \cong K \ltimes N
\end{equation}

where:
\begin{itemize}
\item $K$ is the gyroautomorphism group (the rotations from UNA, isomorphic to $\mathrm{SU}(2)$)
\item $N$ is a normal abelian subgroup (the translations)
\item The action of $K$ on $N$ is by automorphism
\end{itemize}

\textit{Step 4: Minimal abelian extension}

The bi-gyrogroup consistency at ONA demands the minimal abelian subgroup $N$ such that:
\begin{itemize}
\item $N$ is normal in $G$
\item $K$ acts on $N$ by automorphism
\item The structure accommodates both left and right gyroassociative laws
\end{itemize}

For $K \cong \mathrm{SU}(2)$ acting on $\mathbb{R}^n$, the minimal dimension satisfying bi-gyrogroup consistency is $n = 3$. This yields:
\begin{equation}
G \cong \mathrm{SU}(2) \ltimes \mathbb{R}^3 \cong \mathrm{SE}(3)
\end{equation}

the Euclidean group in three dimensions.

\textit{Step 5: Parameter counting}

$\mathrm{SU}(2)$ contributes 3 parameters (rotations).
$\mathbb{R}^3$ contributes 3 parameters (translations).
Total: 6 degrees of freedom.

The semidirect product structure is minimal: fewer translational parameters would not provide sufficient freedom for bi-gyrogroup consistency; more parameters would violate minimality.
\end{proof}

\subsection{Theorem: Non-Existence for $n \neq 3$}

\begin{theorem}[Unique Dimensionality]
The axioms CS1--CS7 are satisfiable if and only if $n = 3$ spatial dimensions.
\end{theorem}

\begin{proof}
We prove this by showing that $n \neq 3$ violates the closure constraint combined with modal depth requirements.

\textbf{Case $n = 2$:}

\textit{Obstruction 1: Rotational insufficiency}

As shown in Lemma 1.2, $n = 2$ admits only $\mathrm{SO}(2) \cong \mathrm{U}(1)$ for rotations, which has one generator. This is insufficient for UNA's gyrocommutativity with CS memory.

\textit{Obstruction 2: Gyrotriangle degeneracy}

In two dimensions, any triangle satisfies $\alpha + \beta + \gamma = \pi$ in Euclidean geometry. However, the gyrotriangle operates in hyperbolic or curved geometry where the defect formula applies. For our specific angles $(\pi/2, \pi/4, \pi/4)$, achieving $\delta = 0$ in 2D would require the triangle to be Euclidean, but this contradicts the non-trivial gyration required by UNA and ONA. The modal depth four requirement (CS3: $\Box B$) cannot be satisfied in 2D with non-trivial gyrations.

\textbf{Case $n = 4$:}

\textit{Obstruction 1: Excess generators}

As shown in Lemma 1.2, $n = 4$ admits $\mathrm{SO}(4)$ with Lie algebra dimension 6. This requires six generators, but only three can be traced to the CS chiral seed (1 DOF). The additional three generators would be independent structure, violating the axiom that ``The Source is Common.''

\textit{Obstruction 2: Gyrotriangle non-closure}

The gyrotriangle defect formula $\delta = \pi - (\pi/2 + \pi/4 + \pi/4) = 0$ achieves exact closure in 3D hyperbolic geometry. In $n \geq 4$ dimensions, the generalized defect formula for hyperbolic $n$-simplices does not reduce to this form. The specific angles $(\pi/2, \pi/4, \pi/4)$ cannot achieve $\delta = 0$ in $n = 4$ while maintaining non-trivial left gyration (CS7), depth-two non-commutation (CS1, CS2), and depth-four commutation (CS3).

\textit{Obstruction 3: Bridge axiom violation}

The bridge axioms CS4 and CS5 connect unity, opposition, and two-step equality. In $n = 4$, the additional generators would create independent paths through the modal space. This would allow configurations where $\Box U$ could hold without forcing $\Box E$ (violating the CS4 constraint structure), or $\Box O$ could hold without forcing $\Box\neg E$ (violating the CS5 constraint structure).

\textbf{Case $n \geq 5$:}

For $n \geq 5$, the dimension of $\mathfrak{so}(n)$ is $n(n-1)/2 \geq 10$. The arguments from the $n = 4$ case apply with even greater force: the excess generators cannot be traced to the CS seed, and the gyrotriangle closure condition cannot be satisfied.

\textbf{Case $n = 3$ (Existence):}

We have shown through Lemmas 1 and 2 that $n = 3$ satisfies all requirements:
\begin{itemize}
\item Exactly 3 rotational generators from $\mathrm{SU}(2)$ (Lemma 1.1)
\item Exactly 3 translational parameters from $\mathbb{R}^3$ (Lemma 2.1)
\item Gyrotriangle closure: $\delta = \pi - (\pi/2 + \pi/4 + \pi/4) = 0$ (verified)
\item Modal depth constraints: CS1--CS7 all satisfied
\end{itemize}
\end{proof}

\begin{corollary}[DOF Progression]
The degrees of freedom emerge in the unique sequence $1 \to 3 \to 6 \to 6(\text{closed})$.
\end{corollary}

\begin{proof}
From the axiom structure:

\textbf{CS (1 DOF):} CS6 and CS7 establish $\mathrm{rgyr} = \mathrm{id}$ and $\mathrm{lgyr} \neq \mathrm{id}$ at the horizon. This asymmetry constitutes exactly 1 degree of freedom (directional distinction). This is the chiral seed.

\textbf{UNA (3 DOF):} Theorem UNA ($\vdash \neg\Box U$) forces $\mathrm{rgyr} \neq \mathrm{id}$. By Lemma 1.1, this requires exactly 3 generators. Total: 3 degrees of freedom (all rotational).

\textbf{ONA (6 DOF):} Theorem ONA ($\vdash \neg\Box O$) forces bi-gyrogroup structure. By Lemma 2.1, this requires exactly 3 translational parameters. Total: $3 + 3 = 6$ degrees of freedom.

\textbf{BU (6 DOF closed):} Theorem BU ($\vdash \Box B$) forces both gyrations to achieve commutative closure at modal depth four. The 6 degrees of freedom remain but become coordinated (no longer independently variable). The system retains complete structural memory while achieving closure.

The progression is unique because each stage follows necessarily from the previous via the axioms, the bridge axioms CS4 and CS5 prevent alternative pathways, and the closure constraint $\delta = 0$ uniquely determines the angles $(\pi/2, \pi/4, \pi/4)$.
\end{proof}

This completes the formal proof that the CGM axioms CS1--CS7 uniquely determine $n = 3$ spatial dimensions with $d = 6$ degrees of freedom.

\newpage

\section{Hilbert Space Representation via GNS Construction}
\label{app:hilbert}

We construct an explicit Hilbert space representation of the Common Governance Model using the Gelfand-Naimark-Segal (GNS) construction. The modal operators $[L]$ and $[R]$ are realized as unitary operators on a Hilbert space $\mathcal{H}$, with the horizon constant $Q_G = 4\pi$ serving as the normalization.

\subsection{The Representation Problem}

The Common Governance Model establishes a formal deductive system with modal operators $[L]$ and $[R]$ satisfying axioms CS1--CS7. To connect this abstract system to quantum mechanics, we require:

\begin{enumerate}
\item A Hilbert space $\mathcal{H}$ (complete inner-product space)
\item Unitary operators implementing $[L]$ and $[R]$
\item Self-adjoint observables corresponding to physical quantities
\item A normalization compatible with $Q_G = 4\pi$
\end{enumerate}

The standard mathematical tool for this is the GNS construction, which produces a Hilbert space representation from any $*$-algebra with a positive linear functional.

\subsection{The $*$-Algebra of Observables}

Define the free $*$-algebra $\mathcal{A}$ generated by two symbols $u_L$ and $u_R$ with relations:

\begin{equation}
u_L u_L^* = u_L^* u_L = I, \quad u_R u_R^* = u_R^* u_R = I
\end{equation}

These encode that $u_L$ and $u_R$ are formal unitaries. We do not initially impose commutativity or any specific product relations between $u_L$ and $u_R$.

The modal operators correspond to:
\begin{itemize}
\item $[L]$: multiplication by $u_L$ on the left
\item $[R]$: multiplication by $u_R$ on the left
\end{itemize}

Composite operators:
\begin{itemize}
\item $[L][R]$: multiplication by $u_L u_R$
\item $[R][L]$: multiplication by $u_R u_L$
\item $[L][R][L][R]$: multiplication by $u_L u_R u_L u_R$
\item $[R][L][R][L]$: multiplication by $u_R u_L u_R u_L$
\end{itemize}

The horizon constant $S$ is represented by the identity element $I \in \mathcal{A}$, scaled by the normalization factor $Q_G = 4\pi$ in the functional $\omega$.

\subsection{The State Functional $\omega$}

Define $\omega: \mathcal{A} \to \mathbb{C}$ as a positive linear functional with:

\begin{equation}
\omega(I) = 1
\end{equation}

The horizon constant normalization is encoded through:

\begin{equation}
\omega(P_S) = Q_G/(4\pi) = 1
\end{equation}

where $P_S$ is the projection onto the horizon sector.

\textbf{Encoding Axiom CS3 (Balance is Universal):}

Axiom CS3 states $\Box B$, meaning the four-step compositions commute. We encode this as:

\begin{equation}
\omega((u_L u_R u_L u_R - u_R u_L u_R u_L)^* (u_L u_R u_L u_R - u_R u_L u_R u_L)) = 0
\end{equation}

This means the difference between the two four-step orderings has zero expectation value in the GNS representation, implementing absoluteness of balance.

\textbf{Encoding CS1 and CS2 (Non-Absoluteness at Depth Two):}

CS1 states $\neg\Box E$ (two-step equality is not absolute). We do not set $u_L u_R = u_R u_L$ as an operator equality. Instead:

\begin{equation}
\omega((u_L u_R - u_R u_L)^* (u_L u_R - u_R u_L)) > 0
\end{equation}

This ensures the two-step compositions differ in the representation, implementing non-absoluteness.

\textbf{Chirality Encoding (CS6, CS7):}

CS6 states $[R]S \leftrightarrow S$, meaning right transition preserves the horizon:

\begin{equation}
\omega(u_R) = \omega(I) = 1
\end{equation}

CS7 states $\neg([L]S \leftrightarrow S)$, meaning left transition alters the horizon:

\begin{equation}
\omega(u_L) \neq \omega(I)
\end{equation}

For concreteness, we can set $\omega(u_L) = e^{i\theta_L}$ with $\theta_L \neq 0$, encoding the chiral phase.

\subsection{GNS Construction}

The GNS construction yields a Hilbert space representation:
\begin{itemize}
\item Define pre-Hilbert space $\mathcal{A}/\mathcal{N}$ where $\mathcal{N} = \{a \in \mathcal{A} : \omega(a^*a) = 0\}$
\item Inner product: $\langle[a], [b]\rangle = \omega(a^* b)$
\item Complete to Hilbert space $\mathcal{H}_\omega$ with cyclic vector $|\Omega\rangle = [I]$
\item Representation: $\pi_\omega(a)[b] = [ab]$, so modal operators are unitary:
\[
U_L = \pi_\omega(u_L), \quad U_R = \pi_\omega(u_R)
\]
\item Satisfies unitarity and $\langle\Omega|U_L U_R U_L U_R|\Omega\rangle = \langle\Omega|U_R U_L U_R U_L|\Omega\rangle$ (CS3)
\end{itemize}


\appendix

\section{Detailed Derivations and Predictions}

\subsection{CGM Duality Parametrisation}

\subsubsection{Topology}
\begin{itemize}
\item BASE: the apex scale (CS)
\item UNION: the "middle" (CS--UNA--ONA)
\item SHELL: the observable shell (BU)
\end{itemize}

\subsubsection{Perspective}
\begin{itemize}
\item BTM (UV-facing, CS focus)
\item TOP (IR-facing, BU focus)
\end{itemize}

\subsubsection{Stages Structure}
\begin{enumerate}
\item CS (Common Source): $s_p = \pi/2$ [dimensionless]
\item UNA (Unity Non-Absolute): $u_p = \cos(\pi/4) = 1/\sqrt{2}$ [dimensionless]
\item ONA (Opposition Non-Absolute): $o_p = \pi/4$ [dimensionless]
\item BU (Balance Universal): $m_p = 1/(2\sqrt{2\pi}) \approx 0.1995$ [dimensionless]
\end{enumerate}

\subsubsection{Geometric Constants}
\begin{itemize}
\item $m_p^2 = 1/(8\pi)$
\item $s_p/m_p^2 = 4\pi^2$
\end{itemize}

\subsubsection{Action Mapping}
\begin{enumerate}
\item $S_{\mathrm{CS}} = s_p / m_p \approx 7.875$
\item $S_{\mathrm{UNA}} = u_p / m_p \approx 3.545$
\item $S_{\mathrm{ONA}} = o_p / m_p \approx 3.937$
\item $S_{\mathrm{BU}} = m_p \approx 0.199$ [identity]
\end{enumerate}

\subsubsection{Union Formation}
\begin{enumerate}
\item $1/S_{\mathrm{UNI}} = \eta/S_{\mathrm{CS}} + 1/S_{\mathrm{UNA}} + 1/S_{\mathrm{ONA}}$
\item For $\eta = 1$: $S_{\mathrm{UNI}} \approx 1.508$, hence $S_{\mathrm{UNI}}/S_{\mathrm{CS}} \approx 0.192$
\end{enumerate}

\subsubsection{Energy Ratios (dimensionless, single-perspective staging)}
\begin{enumerate}
\item $E_{\mathrm{UNA}}/E_{\mathrm{CS}} = 0.450158$
\item $E_{\mathrm{ONA}}/E_{\mathrm{CS}} = 0.500000$
\item $E_{\mathrm{BU}}/E_{\mathrm{CS}} = 0.025330$
\item $E_{\mathrm{UNI}}/E_{\mathrm{CS}} = 0.191518$
\end{enumerate}

\subsubsection{UV Focus Scales (anchor CS at the Planck scale; BTM ≡ UV)}
\begin{enumerate}
\item $E_{\mathrm{CS}}^{\mathrm{BTM}} = 1.22 \times 10^{19}$ GeV
\begin{itemize}
\item $E_{\mathrm{UNA}}^{\mathrm{BTM}} = 5.50 \times 10^{18}$ GeV
\item $E_{\mathrm{ONA}}^{\mathrm{BTM}} = 6.10 \times 10^{18}$ GeV
\item $E_{\mathrm{BU}}^{\mathrm{BTM}} = 3.09 \times 10^{17}$ GeV
\end{itemize}
\item $E_{\mathrm{UNI}}^{\mathrm{BTM}} = 2.34 \times 10^{18}$ GeV
\end{enumerate}

\subsubsection{Optical Conjugacy}
\begin{enumerate}
\item Invariant (for internal stages $i \in \{\mathrm{CS}, \mathrm{UNA}, \mathrm{ONA}, \mathrm{UNI}\}$; BU is the shell anchor):
\[
E_i^{\mathrm{BTM}} \times E_i^{\mathrm{TOP}} = \frac{E_{\mathrm{CS}}^{\mathrm{BTM}} \times E_{\mathrm{BU}}^{\mathrm{TOP}}}{4\pi^2}
\]
\item Anchors: $E_{\mathrm{CS}}^{\mathrm{BTM}} = 1.22 \times 10^{19}$ GeV, $E_{\mathrm{BU}}^{\mathrm{TOP}} = 240$ GeV
\item Invariant $K = 7.42 \times 10^{19}$ GeV²
\end{enumerate}

\subsubsection{IR Focus Scales (TOP ≡ IR; from $E_i^{\mathrm{TOP}} = K / E_i^{\mathrm{BTM}}$)}
\begin{enumerate}
\item $E_{\mathrm{CS}}^{\mathrm{IR}} = 6.08$ GeV
\begin{itemize}
\item $E_{\mathrm{UNA}}^{\mathrm{IR}} = 13.5$ GeV
\item $E_{\mathrm{ONA}}^{\mathrm{IR}} = 12.2$ GeV
\item $E_{\mathrm{BU}}^{\mathrm{IR}} = 240$ GeV (shell anchor at TOP)
\end{itemize}
\item $E_{\mathrm{UNI}}^{\mathrm{IR}} = 31.7$ GeV
\end{enumerate}

\subsubsection{Union Derivation}
\begin{enumerate}
\item Inversion map: $E_i^{\mathrm{BTM}} = K / E_i^{\mathrm{TOP}}$ with $K = (E_{\mathrm{CS}}^{\mathrm{BTM}} \times E_{\mathrm{BU}}^{\mathrm{TOP}}) / (4\pi^2)$
\item Fixed point (conjugacy centre): $E_{\mathrm{MID}} = \sqrt{K} \approx 8.6 \times 10^9$ GeV
\end{enumerate}

\subsubsection{Invariant (equivalent forms)}
\begin{enumerate}
\item Stage-independent product: $E_i^{\mathrm{BTM}} \times E_i^{\mathrm{TOP}} = (E_{\mathrm{BASE}}^{\mathrm{BTM}} \times E_{\mathrm{SHELL}}^{\mathrm{TOP}}) / (s_p/m_p^2)$, with BASE = CS, SHELL = BU
\item Normalised (cross-anchors): $(E_i^{\mathrm{BTM}} / E_{\mathrm{CS}}^{\mathrm{BTM}}) \times (E_i^{\mathrm{TOP}} / E_{\mathrm{BU}}^{\mathrm{TOP}}) = 1 / (4\pi^2)$
\item Centred (via fixed point): $(E_i^{\mathrm{BTM}} / E_{\mathrm{MID}}) \times (E_i^{\mathrm{TOP}} / E_{\mathrm{MID}}) = 1$
\end{enumerate}

\subsubsection{Centre Parametrisation}
Define a dimensionless placement for each stage $i$ relative to the fixed point:
\[
\rho_i = E_i^{\mathrm{TOP}} / E_{\mathrm{MID}}
\]
Then automatically:
\[
E_i^{\mathrm{BTM}} = E_{\mathrm{MID}} / \rho_i
\]
The product $E_i^{\mathrm{TOP}} \times E_i^{\mathrm{BTM}} = E_{\mathrm{MID}}^2$ is independent of $i$. Moving a stage up at TOP by a factor $\rho_i$ moves it down at BTM by the reciprocal factor.

\subsection{Neutrino Mass Derivation}

The framework derives neutrino masses through $48^2$ quantization at the GUT scale:

\begin{equation}
M_R = \frac{E_{\mathrm{GUT}}}{48^2} = 1.01 \times 10^{15} \, \mathrm{GeV}
\end{equation}

Using the type-I seesaw mechanism:

\begin{equation}
m_\nu = \frac{y^2 v^2}{M_R} \approx 0.06 \, \mathrm{eV}
\end{equation}

where $y \approx 1$ is the Yukawa coupling and $v = 246$ GeV is the Higgs vacuum expectation value.

\subsubsection{The Non-Observability of Sterile Neutrinos}

Within the CGM framework, sterile neutrinos with Majorana masses $M_R \approx 10^{15}$ GeV belong fundamentally to the CS (Common Source) domain. CS itself is three-fold, as chirality requires a minimum of three reference points to establish non-commutativity. This three-fold structure at CS manifests as the three sterile neutrino families.

The CS domain is unobservable by principle. As the UV focus of the optical conjugacy relation, CS cannot manifest directly in observations, which occur at the BU (IR) focus. The sterile neutrinos therefore exist entirely within the unobservable CS domain, never appearing at UNA, ONA, or BU stages.

The framework makes a precise distinction:

\textbf{Direct observation impossible:} The framework predicts sterile neutrinos cannot be detected as propagating particles at any observable stage. They remain forever confined to the unobservable CS focus.

\textbf{Indirect effects observable:} Their presence manifests only through:
\begin{itemize}
\item Generation of light neutrino masses via the seesaw mechanism
\item Potential gravitational imprints from a primordial sterile neutrino background
\item The effective Weinberg operator $(LH)(LH)/M_R$ appearing at low energies
\end{itemize}

The apparent tension between CS left-bias and right-handed sterile neutrinos resolves through understanding the two-focus structure:

\begin{itemize}
\item \textbf{CS left-bias:} Determines the order of stage evolution (CS→UNA→ONA→BU) and which focus becomes observable (BU)
\item \textbf{Right-handed neutrinos:} Are SU(2)$_R$ states residing at the CS focus, consistent with CS hosting the unobservable complement to observable left-handed states
\item \textbf{Three-fold CS:} The three sterile families reflect the fundamental three-fold nature of CS required for chirality
\end{itemize}

The CS left-bias does not mean all particles at CS are left-handed. Rather, it ensures that right-handed states at CS remain unobservable while left-handed states at BU become observable.

\subsubsection{Cosmological Implications}

The framework hypothesizes a primordial sterile neutrino background residing at the CS focus. This background:
\begin{itemize}
\item Predates the cosmic microwave background
\item Interacts with the observable sector only gravitationally
\item Cannot be directly detected by any experiment
\item May influence large-scale structure formation through gravitational effects alone
\end{itemize}

\subsubsection{Experimental Predictions}

This principle yields unambiguous predictions:
\begin{enumerate}
\item \textbf{Null results guaranteed:} All direct searches for sterile neutrinos must yield null results, regardless of energy scale or experimental technique.
\item \textbf{Falsification criterion:} Any direct observation of sterile neutrinos as propagating particles would immediately falsify the CGM framework.
\item \textbf{Indirect signatures only:} Sterile neutrino effects can only appear through intermediate mechanisms, never through direct detection.
\end{enumerate}

The continuing null results from experiments worldwide support this geometric principle. These non-observations constitute positive evidence for the structural confinement of sterile neutrinos to the unobservable CS focus.

\section{The Fine-Structure Constant Derivation}

\subsection{UV-IR Foci Structure}

The CGM framework establishes:
\begin{itemize}
\item \textbf{CS (Common Source):} UV focus, unobservable, hosts high-energy physics
\item \textbf{BU (Balance Universal):} IR focus, observable, hosts electromagnetic phenomena
\item \textbf{Optical Conjugacy:} $E_i^{\mathrm{UV}} \times E_i^{\mathrm{IR}} = (E_{\mathrm{CS}} \times E_{\mathrm{EW}})/(4\pi^2)$
\end{itemize}

The fine-structure constant characterizes electromagnetic coupling at the observable BU focus.

\subsection{Base Formula at IR Focus}

The fundamental expression for $\alpha$ at the BU focus is:

\begin{equation}
\alpha_0 = \delta_{\mathrm{BU}}^4 / m_p \tag{1}
\end{equation}

where:
\begin{itemize}
\item $\delta_{\mathrm{BU}} = 0.195342176580$ rad is the BU dual-pole monodromy (measured)
\item $m_p = 1/(2\sqrt{2\pi}) = 0.199471140201$ is the observational aperture parameter (exact)
\end{itemize}

The quartic scaling reflects electromagnetic interaction geometry, while normalization by $m_p$ ensures observational coherence. This yields $\alpha_0 = 0.007299683322$, differing from experiment by 319.398 ppm.

\subsection{Aperture Structure}

The system maintains 97.93\% closure with 2.07\% aperture:

\begin{equation}
\Delta = 1 - \delta_{\mathrm{BU}}/m_p = 0.020699553913 \tag{2}
\end{equation}

This aperture gap enables observation and serves as the expansion parameter for corrections.

\subsection{Systematic Corrections via Foci Transport}

\subsubsection{UV-IR Curvature Correction}

The first correction accounts for curvature between UV and IR foci:

\begin{equation}
\alpha_1 = \alpha_0 \times [1 - (3/4) R \Delta^2] \tag{3}
\end{equation}

where:
\begin{itemize}
\item 3/4 is the exact SU(2) Casimir invariant
\item $R = 0.993434896272$ is the measured Thomas-Wigner curvature ratio
\item $\Delta^2$ represents quadratic aperture effects
\end{itemize}

The curvature $R = (\overline{F}/\pi)/m_p$ with $\overline{F} = 0.622543$ measured at canonical thresholds. This correction captures how geometric transport from UV to IR focus modifies the coupling. Error reduces from 319.398 ppm to 0.052 ppm.

\subsubsection{Holonomy Transport UV→IR}

The second correction encodes holonomy mapping between foci:

\begin{equation}
\alpha_2 = \alpha_1 \times \left[1 - (5/6) \left((\phi_{\mathrm{SU2}}/(3\delta_{\mathrm{BU}})) - 1\right) (1 - \Delta^2 h_{\mathrm{ratio}}) \Delta^2 / (4\pi \sqrt{3})\right] \tag{4}
\end{equation}

where:
\begin{itemize}
\item 5/6: Z$_6$ rotor with one leg open (aperture)
\item $\phi_{\mathrm{SU2}} = 2\arccos((1 + 2\sqrt{2})/4)$: exact SU(2) holonomy
\item $h_{\mathrm{ratio}} = 4.417034$: measured 4-leg/8-leg holonomy ratio
\item 4π: complete solid angle (Q$_G$)
\item √3: 120° rotor geometry projection factor
\end{itemize}

This term captures how UV holonomy structure manifests at the IR focus through geometric projection. Error reduces to -0.000379 ppm.

\subsubsection{IR Focus Alignment}

The final correction aligns residual mismatch at the IR focus:

\begin{equation}
\alpha_3 = \alpha_2 \times [1 + (1/\rho) \mathrm{diff} \, \Delta^4] \tag{5}
\end{equation}

where:
\begin{itemize}
\item $\rho = \delta_{\mathrm{BU}}/m_p = 0.979300$: closure fraction
\item $\mathrm{diff} = \phi_{\mathrm{SU2}} - 3\delta_{\mathrm{BU}} = 0.001874$: monodromic residue
\item $\Delta^4$: fourth-order suppression
\end{itemize}

This ensures coherence at the observable focus after UV-IR transport. Final error: 0.043 ppb.

\subsection{Complete Formula and Results}

The complete formula incorporating all foci corrections:

\begin{align}
\alpha = & (\delta_{\mathrm{BU}}^4 / m_p) \times [1 - (3/4) R \Delta^2] \times \left[1 - (5/6) \left((\phi_{\mathrm{SU2}}/(3\delta_{\mathrm{BU}})) - 1\right) (1 - \Delta^2 h_{\mathrm{ratio}}) \Delta^2 / (4\pi \sqrt{3})\right] \tag{6} \\
& \times [1 + (1/\rho) \mathrm{diff} \, \Delta^4] \nonumber
\end{align}

\textbf{Results:}
\begin{itemize}
\item CGM prediction: $\alpha = 0.007297352563$
\item Experimental value: $\alpha = 0.007297352563$
\item Error: 0.043 ppb (0.532 × experimental uncertainty)
\end{itemize}

\textbf{Error reduction sequence:}
\begin{itemize}
\item Base (IR focus): 319.398 ppm
\item After UV-IR curvature: 0.052 ppm
\item After holonomy transport: -0.000379 ppm
\item After IR alignment: 0.043 ppb
\end{itemize}

\subsection{Physical Interpretation}

The derivation reveals $\alpha$ as emerging from the UV-IR foci structure:

\begin{enumerate}
\item \textbf{IR Focus Geometry:} Base term $\delta_{\mathrm{BU}}^4/m_p$ represents pure electromagnetic coupling at the observable BU focus.

\item \textbf{UV-IR Transport:} Curvature correction accounts for geometric transport between unobservable UV (CS) and observable IR (BU) foci.

\item \textbf{Holonomy Mapping:} Holographic factor encodes how UV holonomy structure projects onto IR observables through 4π solid angle.

\item \textbf{Focus Coherence:} Final correction ensures geometric coherence at the IR focus after incorporating UV influences.
\end{enumerate}

Within CGM, the value 1/137.036 thus emerges from the geometric requirements for electromagnetic phenomena to manifest at the observable focus while maintaining consistency with the unobservable UV origin.

\subsection{Validation}

The derivation's validity rests on:

\begin{enumerate}
\item \textbf{Measured parameters:} All values are measured from CGM geometry, not fitted to $\alpha$
\item \textbf{Systematic convergence:} Each correction reduces error by orders of magnitude
\item \textbf{Foci consistency:} Corrections follow UV→IR transport logic
\item \textbf{No free parameters:} Complete determination from geometric structure
\end{enumerate}


\section{Sterile Neutrinos and the CS Focus}

\subsection{Theoretical Framework}

Within the CGM framework, sterile neutrinos with Majorana masses $M_R \approx 10^{15}$ GeV belong fundamentally to the CS (Common Source) domain. The CS domain is unobservable by principle, as it constitutes the UV focus of the optical conjugacy relation.

\subsection{Direct vs. Indirect Observability}

The framework makes a precise distinction:

\textbf{Direct observation impossible:} Sterile neutrinos cannot be detected as propagating particles at any observable stage.

\textbf{Indirect effects observable:} Their presence manifests only through:
\begin{itemize}
\item Generation of light neutrino masses via the seesaw mechanism
\item Potential gravitational imprints from a primordial sterile neutrino background
\item The effective Weinberg operator $(LH)(LH)/M_R$ appearing at low energies
\end{itemize}

\subsection{Cosmological Implications}

The framework hypothesizes a primordial sterile neutrino background residing at the CS focus. This background predates the cosmic microwave background and interacts with the observable sector only gravitationally.

\subsection{Experimental Predictions}

This principle yields unambiguous predictions:
\begin{enumerate}
\item All direct searches for sterile neutrinos must yield null results
\item Any direct observation of sterile neutrinos as propagating particles would falsify the CGM framework
\item Sterile neutrino effects can only appear through intermediate mechanisms
\end{enumerate}

\section{Electromagnetic Duality Angle}

The ratio of ONA to UNA actions yields the electromagnetic duality angle:

\begin{equation}
\theta = \arctan\left(\frac{S_{\mathrm{ONA}}}{S_{\mathrm{UNA}}}\right) = 48.0^\circ
\end{equation}

This angle characterizes the electromagnetic duality rotation in the framework and appears in the $48^2$ quantization factor for neutrino masses.

\subsection{48° Angle in Geometric Closure}

This analysis explores the relationship between geometric closure principles and the Common Governance Model (CGM) aperture structure through angular harmonics. We demonstrate that the transition from 45° (perfect closure) to 48° (aperture closure) reveals fundamental geometric principles that mirror CGM's structural aperture requirements. The 3° aperture gap ($\pi/60$) creates a 30-fold division of the right angle, establishing harmonic relationships that connect geometric closure to physical observation principles.

The 48° angle introduces intentional geometric aperture:
\begin{itemize}
\item The 3° aperture ($\pi/60$) creates specific geometric constraints for angular momentum
\item The 48° angle may represent optimal conditions for spin system dynamics
\item The 48° angle represents a geometric manifestation of CGM's closure-with-aperture principle
\end{itemize}

The 3° aperture ($\pi/60$) creates:
\begin{itemize}
\item 16:15 harmonic ratio suggesting deep connections between geometric structure and physical dynamics
\item Connection to the 48° angle as a fundamental geometric manifestation of CGM's aperture principle
\end{itemize}

This analysis demonstrates that geometric closure principles operate at multiple scales, with the 48° angle representing a fundamental geometric manifestation of CGM's aperture principle. The exact $\pi/60$ relationship and 16:15 harmonic ratio suggest deep connections between geometric structure and physical dynamics. 